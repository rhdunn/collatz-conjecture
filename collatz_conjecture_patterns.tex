\documentclass{article}
\usepackage[pdfusetitle]{hyperref}
\usepackage{amssymb}
\usepackage{amsmath}
\hypersetup{
	pdfkeywords = {collatz conjecture},
	pdfsubject  = {Collatz conjecture}
}
\title{Patterns in the Collatz Conjecture}
\date{10 August, 2016}
\author{Reece H. Dunn}

\begin{document}

\maketitle

\section{The Collatz Conjecture}

The Collatz conjecture defines the following function:

\begin{displaymath}
f(x) = \begin{cases}
  3x+1        &\text{if } x \text{ is odd} \\
  \frac{x}{2} &\text{if } x \text{ is even} \
\end{cases}
\end{displaymath}

\noindent
This results in a trivial loop \begin{math}[1, 4, 2]\end{math}. The
Collatz conjecture states that no other loops exist and that all
integers greater than zero will produce a sequence that terminates
in the trivial loop.

\section{Even Numbers}

If you can prove any odd number \begin{math}n\end{math} obeys the
Collatz conjecture, then the numbers

\begin{displaymath}
2^{k}n \forall k \in \mathbb{N}
\end{displaymath}

\noindent
also obey the Collatz conjecture. This is because \begin{math}2^{k}n\end{math}
is even, resulting in:

\begin{displaymath}
f(2^{k}n) = \begin{cases}
  f(\frac{2^{k}n}{2}) = f(2^{k-1}n) &\text{if } k>0 \\
  f(n)                              &\text{otherwise} \
\end{cases}
\end{displaymath}

\section{Odd Numbers}

For each odd number \begin{math}n\end{math}, we can find the next odd
number \begin{math}m\end{math} that it maps to. That is, we can find
an odd number \begin{math}m\end{math} that satisfies the equation:

\begin{displaymath}
\exists k \in \mathbb{N} : 3n+1 = 2^{k}m \implies m = \frac{3n+1}{2^{k}}
\end{displaymath}

\end{document}
